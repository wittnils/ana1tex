\section{Differenzierbarkeit}
Es sei $D\subseteq \R^n$ offen und $x_0\in D$. Für eine Funktion $g:D\setminus {x_0}\to \R$ schreiben wir
\[
\lim_{x\to x_0} g(x) = c\in \R,
\]
falls gilt:
\[
\forall \varepsilon>0\ \exists \delta > 0\ \forall x\in D\setminus {x_0}:\ \vert x-x_0 \vert <\delta \implies \vert g(x)-c \vert<\varepsilon.
\]
Äquivalent ist: Für alle Folgen $(x_n)_{n\in\N}$ in $D\setminus{x_0}$ mit $x_n\to x_0$ gilt $g(x_n)\to c$.

\begin{bem}
    Allgemein:
    $G:D\to \R$, setzte $g\coloneqq \left. G\right|_{D\setminus {x_0}}:D\setminus {x_0}\to \R$. Dann ist
    $\lim_{x\to x_0}g(x)=G(x_0) \iff G$ stetig in $x_0$.
\end{bem}
\subsection{Differenzierbarkeit}
\begin{defn} Es sei $I\subseteq \R$ ein Intervall und $f:I\to \R$ eine Funktion. Dann heißt $f$ differenzierbar in $x_0\in I$,
    falls
    \[
        \lim_{x\to x_0} \frac{f(x)-f(x_0)}{x-x_0}
    \] existiert.\\
    Dann heißt
    \[
        f'(x_0)\coloneqq \frac{\D f(x_0)}{\D x}\coloneqq \lim \frac{f(x)-f(x_0)}{x-x_0}
    \] die Ableitung von $f$ in $x_0$.\\
    $f$ heißt differenzierbar auf $I$, falls $f$ in jedem Punkt von $I$ differenzierbar ist. Die Funktion $f':I\to\R, x\mapsto f'(x)$
    ist dann die Ableitung von $f$.
\end{defn}
\begin{lemma} $f:I\to\R$ ist genau dann in $x_0\in I$ diff'bar, wenn es eine Funktion $\Delta: I\to \R$ gibt, die
    \begin{enumerate}[a]
        \item $\Delta$ ist stetig in $x_0$ und
        \item $f(x)=f(x_0)+(x-x_0)\Delta, \forall x\in I$
    \end{enumerate}
    erfüllt. Dann gilt $\Delta(x_0) = f'(x_0)$.
\end{lemma}
\begin{satz} Es sei $f:I\to \R$ diff'bar in $x_0$. Dann ist $f$ auch stetig in $x_0$.
\end{satz}

\subsection{Differentiationsregeln}
Wir nehmen an, dass $f, g: I\to\R$ diff'bar in $x_0\in I$ sind. Dann gilt:
\[
    f(x) = f(x_0)+(x-x_0)\Delta_f(x),\\
    g(x) = g(x_0)+(x-x_0)\Delta_g(x),
\] wobei $\Delta_f, \Delta_g: I\to \R$ stetig in $x_0$.

\begin{satz} Die Funktion $\lambda f, \lambda\in\R$ und $f+g$ sind in $x_0$ diff'bar mit
    \[
        (\lambda f)'(x_0) = \lambda f'(x_0)
    \] und
    \[
        (f+g)'(x_0) = f'(x_0)+g'(x_0).
    \]
\end{satz}
\begin{bem} D.h, der Raum der Funktionen, die in $x_0$ diff'bar sind, ist ein $\R$-Vektorraum und $f\mapsto f'$ ist eine lineare
    Abbildung.
\end{bem}

\begin{satz}{Produkt- bzw. Leibnizregel}
    $f\cdot g$ ist in $x_0$ diff'bar mit
    \[
        (fg)'(x_0) = f'(x_0)g(x_0)+f(x_0)g'(x_0).
    \]
\end{satz}
\begin{satz}{Quotientenregel}
    Es sei $g(x_0)\neq 0$. Dann ist $g(x)\neq 0$ für alle $x$ in der Nähe von $x_0$ und $\frac{f(x)}{g(x)}$ ist in $x_0$ diff'bar mit
    \[
        \left(\frac{f}{g}\right)'(x_0)=\frac{f'(x_0)g(x_0)-f(x_0)g'(x_0)}{(g(x_0))^2}.
    \]
\end{satz}
\begin{satz}{Kettenregel}
    Es seien $f:I\to \R$ und $g:J\to \R$ Funktionen auf Intervallen $I,J\subseteq \R$ mit $f(x)\subseteq J$.
    Falls $f$ in $x_0\in I$ und $g$ in $y_{0}\coloneq f(x_0)\in J$ diff'bar ist, so ist auch $g\circ f$ in $x_0\in I$ diff'bar mit
    \[
        (g\circ f)'(x_0) = g'(f(x_0)\cdot f'(x_0).
    \]
\end{satz}

\begin{satz}{Ableitung der Umkehrfunktion}
    Es sei $f:[a,b]\to \R$ streng monoton, stetig und in $x_0\in [a,b]$ diff'bar mit $f'(x_0)\neq 0$. Dann ist die wohldefinierte Umkehrabbildung (s.o in Topologie)
    $f^{-1}$ in $y_0\coloneqq f(x_0)\in f([a,b])$ ebenfalls diff'bar mit
    \[
        (f^{-1})'(y_0)=(f^{-1})'(f(x_0))=\frac{1}{f'(x_0)}=\frac{1}{f'(f^{-1}(y_0))}.
    \]
\end{satz}