\section{Stetige Funktionen und die Topologie des $\R^n$}
\begin{defn}
    Es sei $\emptyset\neq D\subset \R^n$. Eine reellwertige bzw. komplexwertige Funktion auf $D$ ist eine Abbildung $f:D\to \R$ bzw. $f:D\to \C$.
    Für zwei Funktionen $f,g:D\to \C$ (oder $\R$) definieren wir für alle $x\in D$
    \begin{align*}
        (f+g)(x) &\coloneqq f(x)+g(x) \\
        (f-g)(x) &\coloneqq f(x)-g(x) \\
        (f\cdot g)(x) &\coloneqq f(x)\cdot g(x) 
    \end{align*}
    ist $g(x)\neq 0$ für alle $x\in D$, so auch 
    \[
    \frac{f}{g}(x) \coloneqq \frac{f(x)}{g(x)}    
    \]
    Da $\C$ (und $\R$) Körper sind, sind diese Konstrutkionen wohldefiniert und wieder Abbildungen $D\to \C$ oder $D\to \R$.
\end{defn}
\begin{defn}[Euklidischer Abstand]
    Seien $x=(x_1,\ldots,x_n)$ und $y=(y_1,\ldots,y_n)$ Punkte des $\R^n$, dann definieren wir 
    \[
    \vert x-y\vert \coloneqq \left(\sum_{i=1}^n (x_i-y_i)^2\right)^{1/2}    
    \]
\end{defn}
\begin{defn}[$\epsilon$-$\delta$-Kriterium für Stetigkeit]
    Sei $D\subset \R^n$. Eine Funktion $f:D\to \C$ (oder $\R$) heißt stetig in $x_0\in D$, falls gilt
    \[
    \forall \epsilon >0 : \exists \delta >0 : (\vert x-x_0\vert<\delta) \implies (\vert f(x)-f(x_0)\vert < \epsilon)    
    \]
    Ist $f$ in jedem Punkt von $D$ stetig, so heißt $f$ stetig auf $D$. Wir schreiben
    \[
    \mathscr{C}^0(D,\C) \coloneqq \{ f:D\to \C \mid f \text{ stetig in } D\}    
    \]
    und analog $\mathscr{C}^0(D,\R)$
\end{defn}
\begin{satz}
    Eine Funktion $f:D\to \C$ ist genau dann in $x^*\in D$ stetig, falls für jede Folge $(x_n)_{n\in \N}$ in $D$ mit $\lim x_n =x$ gilt:
    \[
    \lim_{n\to \infty} f(x_n) = f(x^*) = f(\lim_{n\to\infty} x_n)
    \]
    Entsprechend \glqq vertauschen stetige Funktionen mit  Limesbildung\grqq.
\end{satz}
\begin{satz}[Rechenregeln]
    Seien $f,g:D\to \C$ stetig in $x^*\in D$. Dann sind auch $f+g$, $f-g$, $f\cdot g$ und falls definiert auch $f/g$ stetig in $x^*$.
\end{satz}
\begin{satz}[Komposition stetiger Funktione]
    Sei $f:D\to \C$ und $g:E\to \C$ mit $\im f \subset E$. Dann gilt: Ist $f$ stetig in $x^*\in D$ und $g$ stetig ind $f(x^*)\in E$, dann ist $g\circ f:D\to \C$ stetig in $x^*$.
\end{satz}
\begin{defn}[Lipschitz-Stetigkeit]
    Sei $f:D\to \C$ eine Funktion. $f$ heißt Lipschitz-stetig mit Lipschitz-Konstante $L\ge 0$, falls gilt
    \[
    \forall x,y\in D: \vert f(x)-f(y)\vert \le L\cdot \vert x-y\vert    
    \]
\end{defn}
\begin{bem}
    Man sieht leicht, dass jede Lipschitz-stetige Funktion stetig ist. Sei $f:D\to \C$ Lipschitz-steig mit Lipschitz-Konstante $L\ge 0$. Sei $x\in D$ beliebig und $\epsilon>0$ beliebig. Setze $\delta = \epsilon/L$, dann gilt für $y\in D$ mit $\vert x-y\vert<\delta$:
    \[
    \vert f(x)-f(y)\vert \le L \cdot \vert x-y\vert \le  L\cdot \frac{\epsilon}{L}=\epsilon
    \]
    also $f$ stetig in $x\in D$. Da $x\in D$ beliebig war, ist $f$ stetig auf $D$.
\end{bem}
\subsection{Offene und abgeschlossene Mengen}
Zu $a\in\R$ und $r\ge0$ sei
\[
    \B_r(p) \coloneqq \{x\in\R^n \mid \vert x - a \vert \le r\}
\]
der (abgeschlossene) Ball um $a$ von Radius $r$.
\begin{defn}[Randpunkt]
    Es sei $M\subseteq\R^n$ eine Menge. Ein Punkt $p\in\R^n$ heißt Randpunkt von $M$, falls gilt:
    \[
        \forall r>0: \B_r(p)\cap M \neq \emptyset \neq \B_r(p)\cap (\R^n\setminus M).
    \]
    Es bezeichnet:
    \[
        \partial M \coloneqq \{\mathrm{Randpunkte\ von\ } M\}.
    \]
\end{defn}
\begin{defn}[offen/ abgeschlossen] Sei $M\subseteq \R^n$ eine Teilmenge.
    \begin{enumerate}
        \item $M$ heißt offen, falls $M\cap \partial M = \emptyset$ gilt.
        \item $M$ heißt abgeschlossen, falls $\partial M \subseteq M$ gilt.
    \end{enumerate}
\end{defn}
\begin{lemma} Sei $M\subseteq \R^n$ eine Teilmenge.
    \begin{enumerate}
        \item $\partial M = \partial (\R^n\setminus M)$
        \item \(
                  M \mathrm{\ offen\ } \iff \R^n\setminus M \mathrm{\ abgeschlossen}\\
                  M \mathrm{\ abgeschlossen\ } \iff \R^n\setminus M \mathrm{\ offen}
        \)
    \end{enumerate}
\end{lemma}
\begin{lemma}
    $M\subseteq \R^n$ ist genau dann offen, wenn gilt:
    \[
        \forall p\in M\ \exists r > 0: \B_r(p) \subseteq M.
    \]
\end{lemma}
\begin{lemma}
    $M\subseteq \R^n$ ist genau dann offen, wenn gilt:
    \[
        \forall p\in M\ \exists r > 0: \oB_r(p) \subseteq M.
    \]
\end{lemma}
\begin{kor}
    Jede offene Menge ist eine Vereinigung offener Bälle.
\end{kor}
\begin{bem}
    \glqq Punkte sind abgeschlossen.\grqq{} \(\forall p\in\R^n: \{p\}\) is abgeschlossen. Das heißt, dass jede Menge eine Vereinigung abgeschlossener Mengen ist!
    \[
        M = \bigcup_{p\in\R^n}\{p\}
    \]
\end{bem}
\begin{lemma}
    \(M\subseteq \R^n\) ist genau dann abgeschlossen, wenn für alle konvergenten Folgen \((a_n)_{n\in \N}\) in M auch \(\lim a_n \in M\) gilt.
\end{lemma}
\begin{satz}
    \begin{enumerate}
        \item Beliebige Vereinigungen offener Mengen sind offen.
        \begin{itemize}
            \item[Version 1:] Es sei $I$ eine (Index-)Menge. Außerdem sei für jedes $i\in I$ eine offene Menge $U_i\subseteq \R^n$ gegeben. Dann gilt:
            \[
                \bigcup_{i\in I} U_i \subseteq R^n \mathrm{\ ist\ offen.}
            \]
            \item[Version 2:] Es sei $\mathscr{U}\subseteq\mathfrak{P}(R^n)$ mit \(\forall U\in \mathscr{U}: U\) ist offen. Dann gilt:
            \[
                \bigcup \mathscr{U} \coloneqq \{x\in\R^n \mid \exists U\in\mathscr{U}: x\in U\}\subseteq \R^n \mathrm{\ offen.}
            \]
        \end{itemize}
        \item Beliebige Durchschnitte abgeschlossener Mengen sind abgeschlossen.
        \begin{itemize}
            \item[Version 1:] Es sei $I$ eine (Index-)Menge. Außerdem sei für jedes $i\in I$ eine abgeschlossene Menge $A_I\subseteq \R^n$ gegeben. Dann gilt:
            \[
                \bigcap_{i\in I} A_i \subseteq R^n \mathrm{\ ist\ abgeschlossen.}
            \]
            \item[Version 2:] Es sei $\mathrm{A}\subseteq\mathfrak{P}(R^n)$ mit \(\forall A\in \mathscr{A}: A\) ist abgeschlossen. Dann gilt:
            \[
                \bigcap \coloneqq \{x\in \R^n \mid \forall A\in\mathscr{A}: x\in A\} \mathrm{ist abgeschlossen.}
            \]
        \end{itemize}

    \end{enumerate}
\end{satz}
\begin{bem}
    \begin{enumerate}
        \item Die jeweilig umgekehrten Aussagen sind falsch:
        \begin{align*}
            \bigcap_{n\in\N} \left(-1-\frac{1}{n},1+\frac{1}{n}\right) &= [-1,1]\\
            \bigcup_{n\in\N} \left(-1+\frac{1}{n},1-\frac{1}{n}\right) &= (-1,1)
        \end{align*}
        \textbf{Aber}: Endliche Schnitte offener Mengen sind offen. Endliche Vereinigungen abgeschlossener Mengen sind abgeschlossen.
        \item Beide Versionen im Satz sind äquivalent:
        \begin{align*}
            \mathrm{V1} \longrightarrow \mathrm{V2}:\quad& U_i, i\in I \longrightarrow \mathscr{U} \coloneqq = \{U_i \mid i\in I\}\\
            \mathrm{V2} \longrightarrow \mathrm{V1}:\quad& \mathscr{U}\subseteq \mathfrak{P}(\R^n) \longrightarrow I \coloneqq \mathscr{U}: \mathscr{U} = \bigcup_{U\in I=\mathscr{U}}\{U\}
        \end{align*}
    \end{enumerate}
\end{bem}
\begin{defn}[Innere/ Abschluss]
    Das Innere $\interior{M}$ einer Menge $M\subseteq R^n$ ist die Menge $\interior{M} \coloneqq M \setminus \partial M$. Der Abschluss $\exterior{M}$ von $M$ ist $\exterior{M}\coloneqq M\cup \partial M$.
\end{defn}