\section{Grundlagen}
    \begin{defn}[Cantor 1895] Eine Menge ist eine Zusammenfassung
        von wohlbestimmten und wohlunterschiedenen Objekten zu einem Ganzen. Die Objekte
        heißen Elemente der Menge. \\
        Das bedeutet, dass ein Element entweder in der Menge liegt oder nicht und dass jedes 
        Element höchstens einmal in einer Menge vorkommt. \\
        Für ein Element $x$, das in einer Menge $M$ liegt, schreiben wir $x\in M$, sonst $x\notin M$. Wir schreiben
        $\emptyset=\{\}$ für die Menge, die kein Element enthält.
    \end{defn}
    \begin{defn}
        Seien $M,N$ Mengen, wir schreiben
        \begin{enumerate}[(a)]
            \item $M\cup N\coloneqq \{x:x\in M \text{ oder } x\in N\}$, das heißt die Vereinigung von $M$ und $N$.
            \item $M\cap N\coloneqq \{ x:x\in M \ \text{und} \ x\in N\}$, das heißt der Durschnitt von $M$ und $N$.
            \item $M\setminus N\coloneqq \{ x: x\in M \text{ und } x\notin N\}$, das heißt die (relative) Differenz von $M$ und $N$.
            \item $M$ und $N$ heißen disjunkt, falls $M\cap N=\emptyset$.
        \end{enumerate}
    \end{defn}
    Wir wollen nun Aussagen formalisieren. Eine Aussage ist ein Sachverhalt, von dem entweder Wahrheit oder Falschheit ausgesagt werden kann. Der Wahrheitswert
    einer zusammgesetzten Aussage ist eindeutig durch die Wahrheitswerte seiner Teilaussagen festgelegt. 
    \begin{defn}[Aussagen] Seien $A,B$ Aussagen. Wir definieren die Verknüpfungen mittels Wahrheitstabellen. 
    \begin{table}[!ht]
        \centering
        \begin{tabular}{ccccc}\toprule
            A & B & A$\land$ B & A$\lor$B & A$\implies$ B \\ \midrule 
            w & w & w & w & w \\ 
            w & f & f & w & f \\
            f & w & f & w & w \\
            f & f & f & f & w \\ \bottomrule
        \end{tabular}
    \end{table}
    \\ Ferner sage wir $A$, $B$ sind äquivalent, in Zeichen $A\Leftrightarrow B$, falls $(A\implies B) \land (B\implies A)$.
    \end{defn}
    \begin{defn}[Teilmengen] $M,N$ Mengen, wir sagen $M$ ist eine Teilmenge von $N$, Notation: $M\subset N$, falls $x\in M\Rightarrow x\in N$.
    \end{defn}
    \begin{defn}[Quantoren] Wir schreiben abkürzend $\forall$ als \glqq für alle\grqq{}, sowie $\exists$ für \glqq es gibt/es existiert\grqq{} und $\exists !$ für \glqq es gibt genau ein\grqq{}.
    \end{defn}
    \begin{defn}
        $M,N$ Mengen, wir definieren das kartesische Produkt als
        \[
        M \times N \coloneqq \{(m,n) : m\in M, n\in N\}    
        \]
    Eine Teilmenge $R\subset M\times N$ nennen wir Relation. Eine Relation $R\subset M\times N$ heißt rechtseindeutig, falls $\forall x\in M$: es existiert höchstens ein $y\in N:(x,y)\in R$.
    Sie heißt linkstotal, falls $\forall x\in M$: es gibt mindestens ein $y\in N:(x,y)\in R$
    \end{defn}
    \begin{defn}[Abbildungen]
        $M,N$ Mengen. Eine Abbildung $f$ zwischen $M,N$ ist eine Zuordnung, die allen Elementen von $M$ genau ein Element von $N$ zuordnet. 
        In anderen Worten können wir jede linkstotale, rechtseindeutige Relation $R\subset M\times N$ als Abbildung auffassen und umgekehrt.
    \end{defn}
    \begin{defn}
        $f:M\to N$ eine Abbildung, wir definieren 
        \begin{enumerate}[(a)]
            \item $\im f = \{ n\in N \mid \exists m\in M : f(m) =n\}\subset N$
            \item $f^{-1}(B) \coloneqq \{x\in M: f(x)\in B\}\subset M$ für alle $B\subset N$
            \item Für $A\subset M$ definieren wir die Einschränkung von $f$ auf $A$, Notation: $f\mid_A$, durch $f\mid_A:A\to N, a\mapsto f(a)$. 
            \item Den Graphen von $f$ definieren wir als $\text{Graph}(f)\coloneqq \{(x,y)\in M\times N\mid y=f(x)\}$
        \end{enumerate}
        Zwei Abbildungen $f,g:M\to N$ heißen gleich, falls $\forall x\in M:f(x)=g(x)$.
    \end{defn}
    \begin{defn}[Komposition]
        $U,V,W$ Mengen, $f:U\to V, g:V\to W$ Abbildungen, wir definieren die Komposition $g\circ f:U\to W$ durch 
        \[
        (g\circ f)(u)\coloneqq  g(f(u)), \ \forall u \in U    
        \]
    \end{defn}
    \begin{lemma}
        Seien $M,N$ Mengen, $f:M\to N$ eine Abbildung. Seien $A,B\subset M$ und $X,Y\subset N$. Es gilt
        \begin{enumerate}[(a)]
            \item $f(A\cup B)=f(A)\cup f(B)$
            \item $f^{-1}(A\cap B) = f^{-1}(A)\cap f^{-1}(B)$
            \item $f^{-1}(A\cup B) = f^{-1}(A)\cup f^{-1}(B)$
        \end{enumerate}
    \end{lemma}
    \begin{defn}[Injektiv, Surjektiv, Bijektiv] $f:M\to N$ eine Abbildung. $f$ heißt 
        \begin{enumerate}
            \item injektiv, falls $\forall x,y\in M: (f(x)=f(y)) \implies (x=y)$.
            \item surjektiv, falls $\forall y\in N : \exists x\in M : f(x)=y$.
            \item bijektiv, falls $f$ surjektiv und injektiv ist.
        \end{enumerate}
    \end{defn}
    \begin{lemma}
        Falls $f:M\to N$ bijektiv, so existiert genau eine Abbildung $g:N\to M$, sodass $(g\circ f)(m)=m, \ \forall m\in M$ und $(f\circ g)(n)=n, \ \forall n\in N$. Dann heißt $g$ die Umkehrabbildung
        von $f$ und wir schreiben $g=f^{-1}$.
    \end{lemma}
    \begin{proof}
        Für alle $y\in N$ existiert genau ein $x_y\in M$, s.d. $f(x_y)=y$. Wir definieren $g:N\to M$ durch $g(y)=x_y$. Diese Abbildung ist wohldefiniert,
        weil jedes $y\in N$ genau ein Urbild $x\in M$ hat. Dieses $g$ erfüllt offenbar die gewünschten Eigenschaften.\\ 
        Sei $g':N\to M$ mit den geforderten Eigenschaften. Sei $y\in N$ beliebig mit Urbild $x_y\in M$. Dann ist 
        \[
        g(y) = g(f(x_y)) = x_y = g'(f(x_y)) = g'(y)  
        \]
    \end{proof}
    \subsection{Vollständige Induktion} Wir gehen nicht auf die Beweismethode ein. 
    \begin{defn}
        Für $n\in \N_0$, $x\in \R$ definieren wir 
        \[
            x^n = \begin{cases}
                1, & n=0 \\ 
                x\cdot x^{n-1}, & n>0 
            \end{cases}
        \]
    \end{defn}
    \begin{defn}
        Seien $n,m\in \Z$ mit $n\le m$ ohne Einschränkung und $a_n,\ldots,a_m\in \R$ wir definieren 
        \[
        \sum_{k=n}^m a_k \coloneqq a_n+\ldots+a_m    
        \]
    \end{defn}
    \begin{lemma}[Geometrische Reihe]
        Sei $x\in \R\setminus \{1\}$, dann gilt 
        \[
        \sum_{k=0}^n x^k = \frac{1-x^{n+1}}{1-x}    
        \]
    \end{lemma}
    \begin{defn}
        Sei $n\in \N$, wir definieren rekursiv $0!\coloneqq 1$ und $n! = n\cdot (n-1)!$, falls $n>0$. \\
        Ferner definieren wir den Binomialkoeffizienten für $n,k\in \N_0$ als
        \[
        \binom{n}{k} \coloneqq \frac{n!}{k!(n-k)!}, \ n\ge k \qquad \binom{n}{k} \coloneqq 0, \ n<k   
        \]
    \end{defn}
    \begin{lemma}
        Für $0<k<n$ und $k,n\in \N$ gilt 
        \[
        \binom{n}{k} = \binom{n-1}{k-1} + \binom{n-1}{k}    
        \]
    \end{lemma}
    \begin{lemma}
        Für $x,y\in \R$ und $n\in \N_0$ gilt
        \[
        (x+y)^n = \sum_{k=0}^n \binom{n}{k}x^k y^{n-k}    
        \]
        ferner gilt 
        \[
        x^n - y^n = (x-y)\cdot \sum_{k=0}^{n-1} x^{n-k-1}y^k    
        \]
    \end{lemma}
    \section{Körper und angeordnete Körper}
    \begin{defn} Ein Körper ist ein Tripel $(K,+,\cdot)$ bestehend aus einer Menge und zwei binären Verknüpfungen $+$ und $\cdot$, s.d. die folgenden Axiome erfüllt sind
        \begin{enumerate}[(1)]
            \item $a+(b+c)= (a+b)+c, \ \forall a,b,c\in K$
            \item $\exists 0\in K : a+0 = a$
            \item $\forall a\in K : \exists b\in K : a+b=0$
            \item $a+b=b+a, \ \forall a,b\in K$
            \item $a(bc) = (ab)c, \ \forall a,b,c\in K$
            \item $\exists 1\in K^\times  = K \setminus\{0\}: a\cdot 1 = a, \ \forall a\in K$
            \item $\forall a\in K^\times:\exists b\in K^\times : ab=1$
        \end{enumerate} 
    \end{defn}
    \begin{defn}
        Ein angeordneter Körper ist ein Körper zusammen mit einer Teilmenge $P\subset K$, die die folgenden Eigenschaften erfüllt
        \begin{enumerate}
            \item[(O1)] Für jedes $a\in K$ gilt genau eine der folgenden Aussagen \begin{enumerate}[(1)]
                \item $a\in P$
                \item $a\notin P$
                \item $-a\in P$
            \end{enumerate}   
            Man nennt diese Eigenschaft auch oft Trichotomie.
            \item[(O2)] Für alle $a,b\in P$ gilt $a+b\in P, \ a\cdot b\in P$
        \end{enumerate}
    \end{defn}

    \begin{lemma}
        Sei $K$ ein Körper, dann gilt $a^2=(-a)^2$.  
    \end{lemma}
    \begin{proof}Sei $a\in K$ beliebig, es ist
    \begin{align*}
        a+(-a) &= 0 \Rightarrow a^2 + a\cdot (-a) = 0 \Rightarrow a^2 = -(a\cdot(-a)) \\ &= (-1)\cdot (a\cdot (-a)) = ((-1)a)(-a)=(-a)^2
    \end{align*}
    \end{proof}
    \begin{lemma}
        In jedem angeordneten Körper $K$ gilt $a^2\ge 0$ für alle $a\in K$.
    \end{lemma}
    \begin{proof}
        Ist $a\ge 0$, so ist $a\cdot a = a^2\ge 0$ ist $a\le0$, so ist $(-a)\ge 0$ und es ist $0\le (-a)^2 = a^2$
    \end{proof}
    \begin{lemma}
        Jeder angeordnete Körper hat unendliche viele Elemente, insbesondere ist die Abbildung $\N\to K, \ n\mapsto n\cdot 1_K$ injektiv.
    \end{lemma}
    \begin{defn}
        $K$ ein angeordneter Körper, für zwei Elemente $a,b\in K$ definieren wir 
        \[
        \max(a,b) \coloneqq \begin{cases}
            a, & a\ge b\\
            b, & \text{sonst}
        \end{cases}    
        \] 
        Ferner definieren wir den Absolutbetrag auf $K$ durch $\vert \cdot \vert:K\to K, \ a\mapsto \vert a\vert \coloneqq \max(a,-a)$.
    \end{defn}
    \begin{lemma}
        Sei $\vert\cdot\vert$ wie oben definiert, dann gilt
        \begin{enumerate}[(a)]
            \item $\forall a\in K: \vert a\vert \ge 0$
            \item $\forall a,b\in K: \vert ab\vert = \vert a\vert \cdot\vert b \vert $
            \item $\forall a,b\in K: \vert a+b\vert \le \vert a \vert + \vert b\vert$
        \end{enumerate}
        Ferner gilt die sogenannte umgekehrte Dreiecksungleichung 
        \[
        \forall x,y\in K : \vert \vert x\vert - \vert y\vert \vert \le \vert x-y\vert     
        \]
    \end{lemma}
    \section{Die reellen Zahlen $\R$} Sei im Folgenden $K$ stets ein angeordneter Körper mit Ordnungsrelation $\le$.
    \begin{defn}
        Eine Teilmenge $M\subset K$ heißt nach oben/unten beschränkt, falls ein $T\in K$ existiert mit $T\ge x,  \forall x\in K$ beziehungsweise, falls ein $T'\in K$ existiert mit $T'\le x,  \forall x\in K$. 
        Eine Teilmenge heißt beschränkt, falls sie nach oben und nach unten beschränkt ist.
        Jedes Element $T$ beziehungsweise $T'$ mit der obigen Eingenschaft heißt obere/untere Schranke von $M$.
    \end{defn}
    \begin{defn}
        Sei $M\subset K$ eine Teilmenge, falls $M$ eine obere Schranke von $M$ enthält, also ein Element $t\in M$, sodass $t\ge x$ für alle $x\in M$, so heißt $t$ das Maximum der Menge $M$.
        Analog definiert man im Falle der Existenz das Minimum einer Teilmenge von $K$. \\ Wir bezeichnen dieses Element dann mit dem Symbol $\max(M)$ beziehungsweise $\min(M)$.
    \end{defn}
    \begin{bem} Das Maximum(/Minimum) ist eindeutig bestimmt, denn seien $m,m'=\max(M)$, so gilt $m\ge m'$ und $m'\ge m$, also $m=m'$. Analog für das Minimum.
    \end{bem}
    \begin{defn}
        Sei $M\subset K$, ein Element $s\in K$ heißt Supremum von $M$, falls 
        \begin{enumerate}[(a)]
            \item $s$ ist obere Schranke von $M$
            \item Für alle oberen Schranken $t\in K$ von $M$ gilt $t\ge s$
        \end{enumerate}
        $s$ ist in einem bestimmten Sinne also die kleinste obere Schranke von $M$. \\ Analog definiert man das Infimum einer nach unten beschränkten Menge $\emptyset \neq M$ als die 
        größte untere Schranke, Notation: $\inf M$.
    \end{defn}
    \begin{bem}
        Das Supremum einer nicht leeren Teilmenge ist eindeutig bestimmt. Es gibt nicht leere Teilmengen von $\Q$, die kein Supremum in $\Q$ haben.
        Etwa $\{x\in \Q: x^2 < 2\}$
    \end{bem}
    \begin{defn}
        Ein angeordneter Körper $K$ heißt supremumsvollständig (oder auch Dedekindvollständig), falls jede nicht-leere, nach oben beschränkte Teilmenge ein Supremum hat.
    \end{defn}
    \begin{bem}
        Bis auf Isomorphie gibt es einen angeordneten Dedekind-vollständigen Körper, nämlich die reellen Zahlen. 
    \end{bem}
    \begin{satz} Die Menge $\N\subset \R$ ist unbeschränkt. D.h. für alle $x\in \R$ positiv existiert ein $n\in \N$ mit $n> x$
    \end{satz}
    \begin{lemma}Es gelten die folgenden Aussagen 
        \begin{enumerate}[(a)]
            \item Sei $x\in \R$ beliebig, dann existiert eine eindeutig bestimmte ganze Zahl $k\in \Z$ mit $k\le x < k+1$.
            \item Für alle $\epsilon >0$ gibt es ein $n\in \N$ mit $0<1/n < \epsilon$
            \item Die Menge $\Q$ liegt dicht in $\R$, d.h. für alle $x\in \R$ und alle $\epsilon >0$ gibt es ein $q\in \Q$ mit $\vert x-q\vert <\epsilon$.
        \end{enumerate}
    \end{lemma}
    \begin{bem}
        Man kann jede reelle Zahl in der Dezimaldarstellung schreiben. 
    \end{bem}
    \begin{defn}
        Sei $M$ eine Menge mit endlich vielen Elementen, dann nennen wir die Anzahl
        der Elemente von $M$ die Mächtigkeit oder Kardinalität von $M$, Notation $\vert M\vert$ oder $\# M$.\\
        Seien $M,N$ endliche Mengen. Gibt es eine Bijektion $\phi:M\to N$, so ist $\# M =\# N$.
    \end{defn}
    \begin{defn}
        Seien nun $M,N$ zwei beliebige Mengen. Wir nennen $M,N$ gleichmächtig, falls es eine Bijektion $M\to N$ gibt. \\
        Sei $A$ eine Menge, exisitert eine Bijektion $\N\to A$, so heißt $A$ abzählbar, andernfalls heißt $A$ überabzählbar.
    \end{defn}
    \begin{bem}
        Die Mengen $2\N$ und $\Z$ sind beispielsweise abzählbar, denn 
        \begin{enumerate}[(a)]
            \item $\N \to 2\N, \ n\mapsto 2n$ ist eine Bijektion.
            \item Man sieht leicht, dass $\N_0=\N\cup \{0\}$ und $\N$ gleichmächtig sind.
            \[
            \N_0 \to \Z, \ n \mapsto 
            \begin{cases}
            \frac{n}{2}, & n \text{ gerade} \\
            -\frac{n+1}{2}, & n \text{ ungerade}    
            \end{cases}     
            \] ist eine Bijektion.
        \end{enumerate}
        Gleichmächtig ist eine Äquivalenzrelation (reflexiv, symmetrisch klar) und weil die Komposition von Bijektionen eine 
        Bijektion ist, ist sie auch transitiv. 
    \end{bem}
    \begin{satz}
        Die Menge $\Q$ ist abzählbar.
    \end{satz}
    \begin{satz}
        Die Menge der reellen Zahlen $\R$ ist überabzählbar.
    \end{satz}
    \begin{lemma}
        Sei $S$ eine Menge mit mindestens zwei Elementen, dann ist $\operatorname{Abb}(\N,S)$ überabzählbar.
    \end{lemma}
    \begin{defn}
        Sei $M$ eine Menge, dann heißt 
        \[
        \mathfrak{P}(M) = 2^M = \{A:A\subset M\}    
        \] die Potenzmenge von $M$. Die Potenzmenge ist also die Menge aller Teilmengen.
    \end{defn}
    \begin{bem}
        Falls $\# M =n < \infty$, so ist $\# \mathfrak{P}(M) = 2^n$. \\
        Es ist $\mathfrak{P}(\N)$ gleichmächtig zu $\R$. \\
        Frage: Gibt es eine Menge, deren Mächtigkeit zwischen der der reellen Zahlen und der natürlichen Zahlen liegt? (Kontinuumshypothese)
    \end{bem}