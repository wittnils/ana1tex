\section*{Suprema und Infima, Beschränktheit}
\begin{defn}
    Eine Menge $M\subset \R^n$ heißt beschränkt, falls es ein $r\in \R^+$ gibt mit $B_r(0)\supset M$. Im Fall $n=1$ heißt das nichts anderes als $\vert x\vert < r$ für alle $x\in M$.
\end{defn}
\begin{defn}[Supremum und Kriterium]
    Sei $\emptyset\neq M\subset \R$ nach oben beschränkt. Dann definieren wir $\sup M$ als die kleinste obere Schranke von $M$. Wir haben
    folgende Äquivalenzen
    \begin{enumerate}[(i)]
        \item $s = \sup M$
        \item $\forall \epsilon > 0: \exists m\in M: s-\epsilon \le m$
        \item $\forall \epsilon > 0: \exists m\in M: s-\epsilon < m$
    \end{enumerate}
\end{defn}
Das Ganze funktioniert für $\inf M$ analog. 
\begin{satz} Der Körper $\R$ ist supremums-vollständig, das heißt, dass jede nicht-leere nach oben beschränkte Menge ein Supremum
    in $\R$ hat.
\end{satz}
\begin{bem}
    $\Q$ ist nicht supremums-vollständig, denn die Menge $\{x\in \Q:x^2<2\}\subset \Q$ hat kein Supremum in $\Q$. 
\end{bem}
\subsection*{Was ist der Unterschied zwischen Maximum und Supremum?}
\begin{defn}[Maximum]
    Sei $M\subset \R$ nicht leer, dann heißt $x\in \R$ Maximum von $M$, falls 
    \begin{enumerate}[(i)]
        \item $x\in M$.
        \item $\forall y\in M: x\ge y$. 
    \end{enumerate}
    Wir schreiben dann $x = \max M$. Für Minimum analog. 
\end{defn}
Nicht jede nach oben beschränkte Menge $M\subset \R$ besitzt ein Maximum. Zum Beispiel betrachte $A\coloneqq \{ 1- 1/n : n\in \N\}\subset \R$. Dann hat $A$ kein Maximum, denn für alle $x\in A$ existiert ein $n\in \N:x=1-1/n$, aber 
\[
x= 1-\frac{1}{n} < 1-\frac{1}{n+1} \in A    
\]
also kein $x\in M$ Maximum von $A$. $A$ besitzt aber ein Supremum, nämlich ist $\sup A =1$, denn: sei $\epsilon > 0$ beliebig, dann gibt es ein $n\in \N$ mit 
\[
\frac{1}{n} < \epsilon    
\]
daher gilt 
\[
1-\frac{1}{n} > 1-\epsilon \iff 1-\frac{1}{n} + \epsilon > 1    
\]
also ist $\sup A=1$. 
\section*{Folgen, Limes superior und Limes inferior}
Anschaulich gesprochen betrachten wir immer eine Menge zusammen mit einem Konvergenzbegriff. Zum Beispiel ist 
die Aussage jede Cauchy-Folge konvergiert als Aussage über $\Q$ falsch über $\R$ jedoch wahr. Man sagt dann $\Q$ ist nicht vollständig und 
$\R$ ist vollständig. 
\begin{defn}[Folge und Konvergenz]
    Eine Abbildung $a:\N\to \C, \ n\mapsto a(n) \eqqcolon a_n$ heißt Folge. Eine Folge heißt komplexwertige
    konvergent mit Grenzwert $a\in \C$ genau dann, wenn 
    \[
    \forall \epsilon >0 : \exists N(\epsilon) : \forall n\ge N(\epsilon) : \vert a_n -a \vert < \epsilon    
    \] 
    Wir schreiben abkürzend $(a_n)_{n\in \N}$ oder auch nur $(a_n)$. Falls $(a_n)$ gegen $a\in \C$ konvergiert, so schreiben wir 
    \[
        \lim_{n\to \infty} a_n \equiv \lim a_n \coloneqq a
    \]
\end{defn}
\begin{defn}
    Sei $(a_n)$ eine Folge in $\C$. $a\in \C$ heißt Häufungspunkt von $(a_n)$, falls 
    \[
    \forall \epsilon>0 : \textrm{es gibt unendlich viele } n\in \N : \vert a_n - a \vert < \epsilon.    
    \] 
\end{defn}
\begin{bem}
    Der Grenzwert einer Folge ist sein (einziger) Häufungspunkt. Es gilt: $a\in \C$ ist Häufungspunkt von $(a_n)$ genau dann, wenn es eine Teilfolge $(a_{n_k})$ gibt mit $\lim_{k\to\infty} a_{n_k} = a$. 
\end{bem}
\begin{defn}[Cauchy-Folge]
    Eine Folge $(a_n)$ in $\C$ heißt Cauchy-Folge genau dann, wenn 
    \[
    \forall \epsilon > 0: \exists N\in \N:\forall n,m \ge N: \vert a_n -a_m \vert <\epsilon    
    \]
\end{defn}
\begin{bem} Jede konvergente Folge ist eine Cauchy-Folge. Die Umkehrung gilt im Allgemeinen nicht, etwa ist die Folge 
    \[
    a_{n+1} \coloneqq \frac{1}{2}\cdot \left( a_n + \frac{2}{a_n}\right), \ n>1 \quad a_0 \coloneqq 2     
    \]
    hat ausschließlich rationale Folgeglieder und ist konvergent, aber konvergiert nicht in $\Q$.
\end{bem}
\begin{lemma}
    Eine monoton fallende/wachsende nach unten/oben beschränkte Folge $(a_n)$ ist konvergent. 
\end{lemma}
\begin{defn}
    Eine Folge $(a_n)$ heißt bestimmt divergent gegen $+\infty$, falls gilt 
    \[
        \forall K \in \R :\exists n_0 \in \N: \forall n\ge n_0 : a_n \ge K
    \]
\end{defn}
Sei $(a_n)$ eine bestimmt gegen $+\infty$ divergente Folge, dann gilt $a_n >0$ für fast alle 
$n\in \N$ und $\lim 1/(a_n) = 0$. Sei andererseits $(b_n)$ eine Nullfolge mit $b_n>0$ für fast alle $n\in \N$, dann 
gilt, dass die Folge $1/(b_n)$ bestimmt gegen $+\infty$ divergiert.  
\begin{defn}
    Sei $(a_n)$ eine beschränkte Folge, d.h. es gibt ein $S\in \R$ mit $\vert a_n\vert < S, \ \forall n\in \N$. Wir definieren
    \begin{align*}
        B_n &\coloneqq \sup\{a_k : k\ge n\} \\
        b_n &\coloneqq \inf\{a_k : k \ge n\}
    \end{align*}
    Dann sind $(B_n)$ und $(b_n)$ monoton fallend/wachsend und beschränkt, daher konvergent und wir schreiben 
    \begin{align*}
        \limsup_{n\to\infty} a_n &\coloneqq \lim_{n\to \infty} B_n \equiv \inf_{n\in \N} \sup_{k\ge n} a_k \\
        \liminf_{n\to \infty} a_n &\coloneqq \lim_{n\to\infty} b_n \equiv \sup_{n\in \N} \inf_{k\ge n} a_k
    \end{align*}
\end{defn}
Nicht jede beschränkte Folge hat einen Grenzwert. Aber jede beschränkte Folge hat einen Limes superior/Limes inferior. 
Sei $(a_n)$ beschränkte Folge. Falls gilt $\limsup a_n = \liminf a_n$, dann ist $(a_n)$ konvergent und es gilt $\lim a_n = \limsup a_n = \liminf a_n$. \\
Man kann zeigen, dass das $\limsup$/$\liminf$ einer beschränkten Folge gerade das Supremum/Infimum seiner Häufungspunkte ist.